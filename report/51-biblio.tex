\addcontentsline{toc}{chapter}{СПИСОК ИСПОЛЬЗОВАННЫХ ИСТОЧНИКОВ}
\begin{thebibliography}{}
	\bibitem{lit1} Starodubtsev I.A., Vasev P.P. Modeling and Visualization of Lava Flows // 2022
	\bibitem{lit2} Д., Роджерс. Алгоритмические основы машинной графики. / Роджерс Д.
	— M.Мир, 1989. — С. 512. 
	\bibitem{lit3} Foster N., Metaxas D. Realistic Animation of Liquids // Graphical Models and Image Processing. - 1996. - №5. - С. 471-483.
	\bibitem{stam}R. Fedkiw, J. Stam, and H. W. Jensen. Visual Simulation of Smoke. In SIGGRAPH 2001 Conference Proceedings, Annual Conference Series , pages 15-22, August 2001
	\bibitem{lit4} Прилепко М.А Математические модели представления компьютерной графики // 2012. - №8
	\bibitem{lit5} Емельянова Т. В., Аминов Л.А., Емельянов В. А. Реализация алгоритма удаления невидимых граней // Актуальные проблемы военно-научных исследований. - 2021. - №2. - С. 37-44.
	\bibitem{lit6} Никулин, Е.А. Компьютерная геометрия и алгоритмы машинной
	графики. – С.Пб: БХВ–Петербург, 2003. – 560с. 
	\bibitem{lit7} Большая китайская энциклопедия // URL: https://www.zgbk.com/ecph/words?SiteID=1\&ID=153502 (дата обращения: 19.02.2025).
	\bibitem{norway} T. Vik, A.C. Elster, T. Hallgren. Real-time visualization of smoke through parallelizations
	https://folk.idi.ntnu.no/elster/pubs/visual_parco_paper04.pdf
		
		series = {Advances in Parallel Computing},
		publisher = {North-Holland},
		volume = {13},
		pages = {371-378},
		year = {2004},
		booktitle = {Parallel Computing},
		issn = {0927-5452},
		doi = {https://doi.org/10.1016/S0927-5452(04)80049-2},
		url = {https://www.sciencedirect.com/science/article/pii/S0927545204800492},
		author = {T. Vik and A.C. Elster and T. Hallgren},
		abstract = {Publisher Summary
			This chapter discusses real-time visualization of smoke through parallelizations. In order to get the best possible performance, both optimizations of the original model and parallelizations are needed. To achieve this, a 2D prototype was developed, analyzed, and profiled in order to pinpoint possible processing hotspots in the algorithm. The chapter discusses benchmarks and analysis of the data-parallel approach, and shows a significant speedup compared to the serial, but also uncovers that one of the CPUs is left idle, while the other is running the OpenGL rendering code. The speedups gained by parallelizing simulation model proved it very suitable for parallelism, but also revealed the memory bus as being a limitation for memory intensive tasks. The pipelined approach presented introduces a technique for dynamic job distribution and pipelining, thus reducing the penalty caused by serial sub-tasks, such as rendering. The technique of pipelining is well-known, especially within computer hardware engineering, but results showed its relevance within the field of computational fluid dynamics (CFD) visualization.}
	}
	\bibitem{Gauss} Russell M. Cummings, William H. Mason, Scott A. Morton, David R. McDaniel. Birmingham Applied Computational Aerodynamics. --- Cambridge University Press,  2015. --- 888с.
\end{thebibliography}
