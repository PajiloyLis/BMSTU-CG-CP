\chapter{Исследовательский раздел}
В данном разделе проведено исследование разработанной программы.

Исследования проводились на машине со следующими характеристиками:
\begin{itemize}[label=---]
	\item процессор Intel(R) Core(TM) i5-10210U, тактовая частота 1.60 ГГц;
	\item оперативная память: 16 ГБ;
	\item операционная система: Ubuntu 22.04.4 LTS.
\end{itemize}

\section{Цель исследования}

Размер ограничивающего столб пепла параллелепипеда и, как следствие, число вокселей напрямую влияет на производительность симуляции. Большинство методов расчета движения газа имеют кубическую сложность. Метод решения СЛАУ также дополнительно требует провести несколько итераций вычисления каждого неизвестного. В своей статье~\cite{stam} Дж. Стэм предлагает производить вычисления параллельно, для увеличения эффективности.

Цель исследования --- определить какую долю от общего времени выполнения алгоритма, занимают основные его части, для дальнейшего эффективного распараллеливания.

\section{Исследование}

Для исследования выделим основные части в алгоритме основанном на уравнениях Навье---Стокса:
\begin{itemize}
	содержимое...
\end{itemize}

\section*{Вывод}
В данной части была описана цель исследования, технические характеристики устройства, были приведены результаты исследования.

Была произведена серия экспериментов, в результате которых было вычислено среднее время визуализации и обновление физики для разного количества вокселей. Также был проведен опрос для выяснения оптимального количества пикселей для реалистичности воздействия ветра на сакуру. Оказалось, что большинство респондентов выбрали вариант с 5000 вокселей.
