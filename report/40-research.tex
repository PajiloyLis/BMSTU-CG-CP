\chapter{Исследовательский раздел}
В данном разделе проведено исследование разработанной программы.

Технические характеристики устройства, на котором было проведено исследование
\begin{itemize}
	\item Процессор: 12th Gen Intel(R) Core(TM) i5-12400F; 
	\item Оперативная память: 32 Гб;
\end{itemize}



\section{Цель исследования}

Количество вокселей, используемых для разбиения кроны сакуры (листья и лепестки), влияет на качество визуализации дерева. Увеличение числа вокселей позволяет добиться большей детализации изображения, но одновременно повышает вычислительные затраты, увеличивает время рендеринга.

Цель исследования — определить оптимальное число вокселей для представления кроны сакуры. Для этого проведён анализ работы алгоритма визуализации при различном количестве вокселей.

\section{Исследование}

Проводилось исследование для количества вокселей от 500 до 20000. Каждый замер производился 10 раз, было взято среднеарифметическое времени рендеринга и обновления физики.
В таблице ~\ref{table:voxel_timing} приведены результаты замеров

\begin{table}[H]
	\centering
	\caption{Среднее время визуализации и физики в зависимости от количества вокселей}
	\begin{tabular}{|c|c|c|}
		\hline
		Количество вокселей & $t_{\text{ср}}$(визуализация) & $t_{\text{ср}}$(физика) \\
		\hline
		500  &  446.00  &  19.25  \\
		1000  &  611.50  &  20.25  \\
		2000  &  722.50  &  20.00  \\
		3000  &  769.75  &  26.25  \\
		4000  &  729.75  &  21.50  \\
		5000  &  853.75  &  23.75  \\
		6000  &  883.00  &  24.00  \\
		7000  &  935.50  &  25.00  \\
		8000  &  990.00  &  24.67  \\
		9000  &  991.00  &  25.33  \\
		10000  &  1072.25  &  26.50  \\
		15000  &  1260.75  &  28.00  \\
		20000  &  1582.75  &  30.00  \\
		\hline
	\end{tabular}
	\label{table:voxel_timing}
\end{table}
Также был произведен опрос для оценки реалистичности движения сакуры на ветру. Респонденты выбирали самый реалистичный, по их мнению, вариант. В опросе приняли участие 17 человек. Результат опроса приведен в таблице~\ref{tab:rate}
\begin{table}[H]
	\centering
	\caption{Результаты опроса}
	\begin{tabular}{|c|c|}
		\hline
		\textbf{Количество вокселей} & \textbf{Количество проголосовавших} \\
		\hline
		500 & 2 \\
		\hline
		1000 & 1 \\
		\hline
		5000 & 5 \\
		\hline
		10000 & 4 \\
		\hline
		15000 & 2 \\
		\hline
		20000 & 3 \\
		\hline
	\end{tabular}
	\label{tab:rate}
\end{table}
В результате выяснилось, что большинство респондентов проголосовало за вариант с 5000 вокселями, что обеспечивает хорошую реалистичность и относительно высокую скорость рендеринга.
\section*{Вывод}
В данной части была описана цель исследования, технические характеристики устройства, были приведены результаты исследования.

Была произведена серия экспериментов, в результате которых было вычислено среднее время визуализации и обновление физики для разного количества вокселей. Также был проведен опрос для выяснения оптимального количества пикселей для реалистичности воздействия ветра на сакуру. Оказалось, что большинство респондентов выбрали вариант с 5000 вокселей.
