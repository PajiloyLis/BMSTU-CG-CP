\ssr{ВВЕДЕНИЕ}

Вулканическая деятельность оказывает существенное влияние на окружающую среду, экосистемы и ландшафты Земли. Компьютерное модели вулканов активно применяется в научных исследованиях, геологии, мониторинге катастроф и образовательных программах. Современные технологии позволяют не только визуализировать вулканическую активность, но и анализировать её влияние на окружающую среду, что открывает новые возможности для исследований и прогнозирования.\cite{lit1}

Цель курсовой работы -- разработка программного обеспечения для визуализации процесса извержения вулкана.
В рамках реализации курсовой работы должны быть решены следующие задачи:
\begin{itemize}
	\item провести анализ алгоритмов  удаления невидимых линий и поверхностей, построения теней и освещения и выбрать наиболее подходящие для решения задачи;
	\item спроектировать программное обеспечение;
	\item выбрать средства реализации программного обеспечения и разработать его;
	\item провести исследование характеристик разработанного программного обеспечения.
\end{itemize}

